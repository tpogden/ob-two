\documentclass{tufte-handout}

%\geometry{showframe}% for debugging purposes -- displays the margins

\geometry{a4paper}

\usepackage{amsmath}

% Set up the images/graphics package
\usepackage{graphicx}
\setkeys{Gin}{width=\linewidth,totalheight=\textheight,keepaspectratio}
\graphicspath{{graphics/}}

\title{Title}
\author[tommyogden]{Tommy Ogden}
\date{4 May 2010}  % if the \date{} command is left out, the current date will be used

% The following package makes prettier tables.  We're all about the bling!
\usepackage{booktabs}

% The units package provides nice, non-stacked fractions and better spacing
% for units.
\usepackage{units}

% The fancyvrb package lets us customize the formatting of verbatim
% environments.  We use a slightly smaller font.
\usepackage{fancyvrb}
\fvset{fontsize=\normalsize}

% Small sections of multiple columns
\usepackage{multicol}

% Provides paragraphs of dummy text
\usepackage{lipsum}

% These commands are used to pretty-print LaTeX commands
\newcommand{\doccmd}[1]{\texttt{\textbackslash#1}}% command name -- adds backslash automatically
\newcommand{\docopt}[1]{\ensuremath{\langle}\textrm{\textit{#1}}\ensuremath{\rangle}}% optional command argument
\newcommand{\docarg}[1]{\textrm{\textit{#1}}}% (required) command argument
\newenvironment{docspec}{\begin{quote}\noindent}{\end{quote}}% command specification environment
\newcommand{\docenv}[1]{\textsf{#1}}% environment name
\newcommand{\docpkg}[1]{\texttt{#1}}% package name
\newcommand{\doccls}[1]{\texttt{#1}}% document class name
\newcommand{\docclsopt}[1]{\texttt{#1}}% document class option name

\begin{document}

\maketitle% this prints the handout title, author, and date

\begin{abstract}
\noindent This is a user guide for OB Two.
\end{abstract}

\include{Introduction}


\begin{align}
    \langle m_{A1} m_{A2} | x'_A x'_b | m_{B1} m_{B2} \rangle
    = \langle m_{A1} | x'_A | m_{A2} \rangle \langle m_{B1} | x'_B | m_{B2} \rangle \\
\end{align}

    %\frac{1}{2} \left[ 

%    \right]

\begin{fullwidth}
\begin{align}
    \langle m_{A1} | x'_A | m_{A2} \rangle = & 
    \frac{1}{2} [ \langle sm_{A1} | x_A' | sm_{A2} \rangle + (-1)^{m_{A2}-\frac{1}{2}} \langle sm_{A1} | x'_A | pm_{A2} \rangle e^{-i \omega t} + \\
    & (-1)^{m_{A1}-\frac{1}{2}} \langle pm_{A1} | x'_A | sm_{A2} \rangle e^{i \omega t} + (-1)^{m_{A1} + m_{A2}-1} \langle pm_{A1} | x_A' | pm_{A2} \rangle ] 
\end{align}

\begin{align}
\frac{1}{4} & [ (-1)^{m_{A2} + m_{B2}-1} \langle s_{A1} | x'_A | p_{A2} \rangle \langle s_{B1} | x'_B | p_{B2} \rangle e^{-2i \omega t} + \\
& (-1)^{m_{A2} + m_{B1}-1} \langle s_{A1} | x'_A | p_{A2} \rangle \langle p_{B1} | x'_B | s_{B2} \rangle + \\
& (-1)^{m_{A1} + m_{B2}-1} \langle p_{A1} | x'_A | s_{A2} \rangle \langle s_{B1} | x'_B | p_{B2} \rangle + \\
& (-1)^{m_{A1} + m_{B1}-1} \langle p_{A1} | x'_A | s_{A2} \rangle \langle p_{B1} | x'_B | s_{B2} \rangle e^{2i \omega t} ] 
\end{align}

\begin{align}
    P_2 = (1 + 3 \cos{2 \gamma})/4
\end{align}

\end{fullwidth}





\end{document}